\section{Conclusion}
    In this experiment, we successfully built a spectrometer using the grating feature of a cd.
    As a first step, we were required to calibrate the spectrometer.
    As we did not have a monochromatic light source at hand, we used a literature value for the grating
    constant $g$. Next, we had to determine the distance $d$ of the camera sensor and the grating.
    We concluded that $\bf d = (1.7 \pm 0.1) ~\si{\bf\milli\meter}$.
    Finally, we were able to analyze some different light sources, to determine their respective wave spectrums.
    We found that some indoor ceiling lights emit only certain wavelengths, while others emit almost the entire
    visible spectrum.
    We are very happy with how our spectrometer turned out. We can calculate the wavelength of any color
    in the detected spectrum, with a satisfactory error margin.

    Unfortunately, we were not able to detect Frauenhofer lines in our measurements taken from sunlight.
    This is due to the fact, that our device is not accurate enough to distinguish these fine lines in the spectrum.
    To improve this result, we would have to make the following changes to our device:
    \begin{itemize}
        \item Improve the consistency in the gap of the razor blades.
        \item Coat the inside of the box with a non-reflective dark paint.
        \item Use a higher-resolution camera, with a manual focus setting.
        \item Build a spectrometer using a grating with a smaller grating constant $g$, as this would lead to the same spectrum split into a bigger area.
            A DVD perhaps even with blu-ray-technology could maybe do the job.
    \end{itemize}

    % In the concluding paragraph you summarize the result, with the
    % emphasis on what you have discovered in this work. You can end this
    % with an outlook on future research, i.e. how could the results be
    % improved or what would be a logical follow-up experiment.