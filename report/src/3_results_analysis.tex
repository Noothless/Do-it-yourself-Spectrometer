\section{Results}\label{sec_results}

As mentioned in the previous chapter, the first goal is to find the distance $d$ of the grating
and the image sensor.
By combining eq. \ref{eq_interference} and eq. \ref{eq_phi} we get the following equation for $d$:

\begin{align}
    d = \frac{x}{\tan(\phi)} = \frac{x}{\tan(\arcsin(\frac{\lambda}{g}))} \label{eq_distance}
\end{align}

Multiple steps are necessary to calculate the distance $d$.
First of all, we took a picture with the spectrometer pointed at the sunlight. We then selected suitable 
points in the spectrum and compared them to tabulated wavelengths and their corresponding colors 
in fig.~\ref{fig_wavelengths}. In the next step we measured the distance of said points to the zero-order maximum
using the imtools method in matlab. This is our pixel distance $x$ in eq.~\ref{eq_distance}.

For our final calculation of the distance $d$ we averaged multiple measurement points to get a more
accurate result. This is shown in fig.~\ref{fig_distance_calc}.

The error calculations for eq. \ref{eq_distance} were done using the gaussian error propagation.
Please refer to our jupyter notebook \cite{GitHub}, for the full calculations of all the values and 
error propagations presented in this chapter. 
\begin{align}
    \Delta d = \sqrt{\left(\frac{\partial d}{\partial x}\right)^2 + \left(\frac{\partial d}{\partial \lambda}\right)^2}
\end{align}

Our averaged result for $d$ is:
\begin{align}
    \bf d = (1.71 \pm 0.12) ~\si{\bf\milli\meter} \label{res_d}
\end{align}

We have now successfully calibrated our spectrometer and we can now determine the wavelength
of any point in the picture.
The following equation is used to determine the wavelength of a point with a certain pixel 
distance $x$ to the zero-order maximum.
\begin{align}
    \lambda = g \sin\left(\arctan\left(\frac{x}{d}\right)\right) \label{eq_lambda}
\end{align}

Again the uncertainties were determined using the gaussian error propagation.
\begin{align}
    \Delta \lambda = \sqrt{\left(\frac{\partial \lambda}{\partial x}\right)^2 + \left(\frac{\partial \lambda}{\partial d}\right)^2}
\end{align}

Using these obtained results we finally analyzed some different types of light
sources.