\section{Results}\label{sec_results}

In our case, we will calculate the grating constant $g$ using hypothetical values for the reasons mentioned in section \textbf{insert section and information from GPT}
By rearranging and combining eq.\ref{eq_interference} and eq.\ref{eq_phi} we find a formula for the grating constant\ref{eq_g}.

\begin{align}
    g &= \frac{p \lambda}{\sin(\phi)} \label{eq_g}\\
    \phi &= \arctan \left( \frac{x}{D} \right)
\end{align}

To calculate the uncertainty of the grating constant, the gaussian error propagation is a useful tool.

\begin{align}
    \Delta g &= \sqrt{\left( \frac{\partial g}{\partial p} \Delta p \right)^2 + \left( \frac{\partial g}{\partial \lambda} \Delta \lambda \right)^2 + \left( \frac{\partial g}{\partial \phi} \Delta \phi \right)^2}\\
    \Delta \phi &= \sqrt{\left( \frac{\partial \phi}{\partial x} \Delta x \right)^2 + \left( \frac{\partial \phi}{\partial D} \Delta D \right)^2}
\end{align}

The hypothetical value of the grating constant $g$ and its uncertainty can now be calculated:

\begin{align}
    \bf g = (1520 \pm 20) ~\si{\bf\nano\meter} = (1.52 \pm 0.02) ~\si{\bf\micro\meter} \label{res_g}
\end{align}

Note that the obtained value is not the real value and that a different value for the grating constant $g$ \cite{src_grating_constant} is used in the remaining calculations.

As mentioned in the previous chapter, the first goal is to find the distance $d$ of the grating
and the image sensor.
By combining eq. \ref{eq_interference} and eq. \ref{eq_phi} we get the following equation for $d$:

\begin{align}
    d = \frac{x}{\tan(\phi)} = \frac{x}{\tan(\arcsin(\frac{\lambda}{g}))} \label{eq_distance}
\end{align}

Multiple steps are necessary to calculate the distance $d$.
First of all, we took a picture with the spectrometer pointed at the sunlight \ref{fig_sunlight}. We then selected suitable 
points in the spectrum and compared them to tabulated wavelengths and their corresponding colors 
in fig.~\ref{fig_wavelengths}. In the next step we measured the distance of said points to the zero-order maximum
using the imtools method in matlab. This is our pixel distance $x$ in eq.~\ref{eq_distance}.

For our final calculation of the distance $d$ we averaged multiple measurement points to get a more
accurate result. This is shown in fig.~\ref{fig_distance_calc}.

The error calculations for eq. \ref{eq_distance} were done using the gaussian error propagation.
Please refer to our jupyter notebook \cite{GitHub}, for the full calculations of all the values and 
error propagations presented in this chapter. 
\begin{align}
    \Delta d = \sqrt{\left(\frac{\partial d}{\partial x}\right)^2 + \left(\frac{\partial d}{\partial \lambda}\right)^2}
\end{align}

Our averaged result for $d$ is:
\begin{align}
    \bf d = (1.71 \pm 0.12) ~\si{\bf\milli\meter} \label{res_d}
\end{align}

We have now successfully calibrated our spectrometer and we can now determine the wavelength
of any point in the picture.
The following equation is used to determine the wavelength of a point with a certain pixel 
distance $x$ to the zero-order maximum.
\begin{align}
    \lambda = g \sin\left(\arctan\left(\frac{x}{d}\right)\right) \label{eq_lambda}
\end{align}

Again the uncertainties were determined using the gaussian error propagation.
\begin{align}
    \Delta \lambda = \sqrt{\left(\frac{\partial \lambda}{\partial x}\right)^2 + \left(\frac{\partial \lambda}{\partial d}\right)^2}
\end{align}

Using these obtained results we finally analyzed some different types of light
sources. In each case, we assumed an error of $\pm 10$ pixels, since we had to guess the position of each measurement point in the picture.

\newpage

First of all, we analyzed a fluorescent tube lamp. As can be seen in fig. \ref{fig_lamp1} below, this type of light source only emits certain small 
ranges of wavelength. In our case, we measured the following ranges:
\begin{alignat}{3}
    &\text{Red spectrum:} \; &&(587 \pm 36)~\si{\nano\meter} & &- (622 \pm 38)~\si{\nano\meter} \nonumber\\
    &\text{Green spectrum:} \; &&(487 \pm 32)~\si{\nano\meter} & &- (560 \pm 35)~\si{\nano\meter} \nonumber\\
    &\text{Blue spectrum:} \; &&(430 \pm 29)~\si{\nano\meter} & &- (455 \pm 31)~\si{\nano\meter} \nonumber
\end{alignat}
\begin{figure}[H]
    \centering
    \includegraphics[scale = 0.31]{src/images/lamp1_meas.png}
    \caption{Spectrum of a fluorescent tube.}
    \label{fig_lamp1}
\end{figure}

As a second example, we analyzed a LED ceiling lamp. As can be seen in fig. \ref{fig_lamp2} below, this lamp
emits almost the entire spectrum of visible light with no gaps visible. We measured the following range:
\begin{alignat}{3}
    &\text{Full spectrum:} \; &&(429 \pm 29)~\si{\nano\meter} & &- (650 \pm 38)~\si{\nano\meter} \nonumber
\end{alignat}
\begin{figure}[H]
    \centering
    \includegraphics[scale = 0.41]{src/images/lamp2_meas.png}
    \caption{Spectrum of an LED ceiling light.}
    \label{fig_lamp2}
\end{figure}

Lastly, we analyzed the computer screen of a laptop. In this example, the color purple was shown on the screen.
As can be seen in fig. \ref{fig_purple_screen} only two very narrow bands of wavelengths are emitted.
\begin{alignat}{3}
    &\text{Red spectrum:} \; &&(597 \pm 37)~\si{\nano\meter} & &- (622 \pm 38)~\si{\nano\meter} \nonumber\\
    &\text{Blue spectrum:} \; &&(433 \pm 29)~\si{\nano\meter} & &- (457 \pm 37)~\si{\nano\meter} \nonumber
\end{alignat}
\begin{figure}[H]
    \centering
    \includegraphics[scale = 0.25]{src/images/purple_screen_meas.png}
    \caption{Spectrum of a computer screen displaying a purple color.}
    \label{fig_purple_screen}
\end{figure}

Helium discharge tube
Quecksilver light, UV A 400nm, UV C 250nm